\documentclass[utf8,a4paper,nofonts,9pt]{ctexbook}

\usepackage{amsmath}
\usepackage{xcolor}
\usepackage{hyperref}
\usepackage{graphicx}
\usepackage[many]{tcolorbox}
\usepackage[left=0.65in, right=0.65in, top=0.95in, bottom=0.95in]{geometry}
\usepackage{enumitem}
\setlist[1]{
    leftmargin=2em,
    labelsep=0.25em,
    labelwidth=0.72em,
    listparindent=2em,
    align=left,
    itemsep=-.1em,
}
\usepackage{src/package/Listings}

\NewDocumentCommand\TODO{m}{\footnote{\textcolor{red}{TODO: #1}}}

\def\dif{\mathop{}\!\mathrm{d}}

\newtcolorbox{exampleBox}{
    sharpish corners, % better drop shadow
    boxrule = 0pt,
    toprule = 4.5pt, % top rule weight
    enhanced,
    fuzzy shadow = {0pt}{-2pt}{-0.5pt}{0.5pt}{black!35} % {xshift}{yshift}{offset}{step}{options} 
}

\setCJKmainfont[Path=../fonts/,BoldFont=SourceHanSerifCN-Bold.otf,ItalicFont=SourceHanSerifCN-Bold.otf]{SourceHanSerifCN-Regular.otf}
\setCJKsansfont[Path=../fonts/,BoldFont=SourceHanSansCN-Bold.otf]{SourceHanSansCN-Regular.otf}
\setCJKmonofont[Path=../fonts/]{SourceHanSansCN-Regular.otf}

\begin{document}

\underline{问题} 对于一个大数据集,比如 256 TiB 的数据集,我们如何使用有限的内存,并且在受限环境(同时读写的文件数受限,能够产生的文件数受限)中完成乱序读取?

\underline{解} 我们首先可以假设我们拥有无限大小的内存,这个时候一个解法就是将数据集加载到内存中,然后使用 $\mathcal{O}(n)$ 的算法打乱,然后读取。

但是我们没有无限大小的内存。后续我们假设应用能使用的内存大小为 16 GiB(即我们如果分批读取的话,需要读取 $16 \times 2^{10}$ 次)。一种解法就是我们可以生成 $16 \times 2^{10}$ 个文件,每个文件的大小为 16 GiB,我们通过如下的算法生成:

\begin{itemize}
    \item 流式读取数据集。
    \item 对于每个记录而言,我们生成一个随机值,并对 $16 \times 2^{10}$ 取模,结果记为 $i$。
    \item 如果记录对应随机值为 $i$,那么将记录写入到第 $i$ 个文件中。
\end{itemize}

当我们流式处理完这个数据集后,我们得到了很多文件,每个文件都可以加载到内存中,同时每个文件都是对整个数据集的抽样(即我们可以说对于某个文件而言,它的每个记录在抽样前均匀随机地位于大数据集之中)。
这个时候,随机读取大数据集,等效于:

\begin{itemize}
    \item 顺序遍历所有文件。
    \item 对于每个文件,我们在读取前,需要先加载在内存中并打乱。然后读取打乱后的记录。
\end{itemize}

但是受到文件系统所限,我们无法让 opened FD 数量过大。我们假设我们能接收的 opened FD 数量为 256。即如果我们打开 256 个文件,每个文件写入 16 GiB 的话,我们一批即可写入 4 TiB,最终我们需要写入 64 批。
我们可以先暂时放开一点要求,即我们能接受文件的记录不是均匀地位于整个大数据中,而是只要位于一个足够大的区间内即可。那么我们可以:

\begin{itemize}
    \item 在逻辑上将大数据集划分为多个区间。这里我们划分为 64 个区间,每个区间大小 4 TiB。
    \item 对于每个区间,我们维护 256 个 opened FD。
    \item 和之前的逻辑一样,我们将区间的记录随机地打散到多个文件中。
\end{itemize}

这样,我们能在受限的 opened FD 约束下完成需求。

\end{document}
